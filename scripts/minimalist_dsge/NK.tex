\documentclass[12pt]{article}

\usepackage{amsmath}
\usepackage{amssymb}
\usepackage{geometry}
\geometry{margin=2cm}

\title{Two Sector New Keynesian Model with Weather Shocks and Land}
\author{}
\date{}

\begin{document}

\maketitle

\section{Households}

A representative household consumes a CES bundle of agricultural and non agricultural
goods and supplies differentiated labor to the two sectors.

Habit adjusted consumption:
\[
\tilde c_t = c_t - b c_{t-1}.
\]

Preferences:
\[
U = \mathbb{E}_0 \sum_{t=0}^{\infty} \beta^t 
\left[
\frac{\tilde c_t^{\,1-\sigma_C}}{1-\sigma_C}
- \chi\, h_{u,t}^{\,1+\sigma_H}
\right],
\]
where the labor aggregator with reallocation costs is
\[
h_{u,t}
= \left(h_{N,t}^{1+\\iota} + h_{A,t}^{1+\\iota}\right)^{\frac{1}{1+\\iota}}.
\]

Marginal utility of consumption and stochastic discount factor:
\[
u_{c,t} = \tilde c_t^{-\sigma_C}, 
\qquad 
m_t = \beta \frac{u_{c,t}}{u_{c,t-1}}.
\]

Intratemporal optimality:
\[
w^N_t\, u_{c,t} = u^N_t, 
\qquad 
w^A_t\, u_{c,t} = u^A_t,
\]
with sector specific marginal disutilities of labor
\[
u^N_t = e^h_t\, \chi\, h_{u,t}^{\sigma_H}\left(\frac{h_{N,t}}{h_{u,t}}\right)^\iota,
\qquad
u^A_t = e^h_t\, \chi\, h_{u,t}^{\sigma_H}\left(\frac{h_{A,t}}{h_{u,t}}\right)^\iota.
\]

Nominal Euler equation (one period nominal bond):
\[
m_{t+1} \frac{R_t}{\pi_{t+1}} = 1,
\]
where $R_t$ is the gross nominal interest rate and $\pi_t$ is gross aggregate inflation.

\section{Production}

The economy has two producing sectors: non agriculture ($N$) and agriculture ($A$).

Production is linear in TFP and concave in labor:
\[
y^N_t = e^z_t\, h_{N,t}^{\,1-\alpha}.
\]

With real marginal cost $mc^N_t$, cost minimization implies:
\[
w^N_t = mc^N_t (1-\alpha)\, p^N_t \frac{y^N_t}{h_{N,t}},
\]
where $p^N_t$ is the relative price of non agricultural goods.

Agricultural sector with land and weather damage. Weather affects agricultural production through a damage function. Let $d_t$ denote the
weather damage factor
\[
d_t = (e^s_t)^{-\theta_1}.
\]

Agricultural output:
\[
y^A_t
= (d_t\,\ell_{t-1})^{\omega}
\left(e^z_t \kappa_A\, h_{A,t}^{\,1-\alpha}\right)^{1-\omega}.
\]

With real marginal cost $mc^A_t$, labor demand in agriculture:
\[
w^A_t 
= mc^A_t (1-\omega)(1-\alpha)\, p^A_t\,\frac{y^A_t}{h_{A,t}},
\]
where $p^A_t$ is the relative price of agricultural goods.

\section{Price setting and nominal rigidities}

Prices are sticky in both sectors. We adopt Rotemberg type quadratic adjustment costs
on inflation. For sector $j \in \{N,A\}$, let $\epsilon_j$ denote the elasticity of substitution
and $\kappa_j$ the adjustment cost parameter.

Denote sectoral inflation by $\pi_{j,t}$ and aggregate inflation by $\pi_t$.
The New Keynesian Phillips curves in each sector take the form
\[
(1-\epsilon_j)\, p^j_t y^j_t 
+ \epsilon_j\, mc^j_t y^j_t
- \kappa_j\, p^j_t \pi_{j,t}(\pi_{j,t}-1)\, y^j_t
+ \kappa_j\, \mathbb{E}_t\left[ m_{t+1} p^j_{t+1} \pi_{j,t+1}(\pi_{j,t+1}-1)\, y^j_{t+1} \right]
= 0,
\]
for $j = N,A$.

\section{Land, land costs, and damage}

Land evolves according to
\[
\ell_t = (1-\delta_L)\,d_t\, \ell_{t-1} + \phi_t,
\]
where $\phi_t$ is land investment.

Land cost function:
\[
\phi_t 
= \frac{\tau}{\psi}\, x_t^{\psi}\, d_t\, \ell_{t-1},
\]
with $x_t$ the land investment intensity.

Shadow value of land $\varrho_t$ satisfies
\[
\varrho_t
=
\mathbb{E}_t\left\{
m_{t+1}
\left[
mc^A_{t+1}\frac{\omega\, p^A_{t+1} y^A_{t+1}}{\ell_t}
+
(1-\delta_L)d_{t+1}\varrho_{t+1}
+
\frac{\phi_{t+1}}{\ell_t}
\right]
\right\}.
\]

FOC for land investment $x_t$:
\[
p^N_t
=
\tau\, x_t^{\psi-1}\, \varrho_t\, d_t\, \ell_{t-1}.
\]

\section{Aggregation, demand, and prices}

Let $n_t$ be the agricultural sector share, driven by a reallocation shock:
\[
n_t = \bar n\, e^n_t.
\]

Aggregate output:
\[
y_t
= (1-n_t)\, p^N_t y^N_t + n_t\, p^A_t y^A_t.
\]

Aggregate hours:
\[
h_t = (1-n_t) h_{N,t} + n_t h_{A,t}.
\]


Non agricultural goods:
\[
(1-n_t) y^N_t
= (1-\varphi)\, (p^N_t)^{-\mu}\, c_t
+ n_t x_t 
+ g_y\, \bar y^N\, e^g_t.
\]

Agricultural goods:
\[
n_t y^A_t
= \varphi\, (p^A_t)^{-\mu}\, c_t.
\]

\subsection{Relative price index}

The CES price index is
\[
1 = (1-\varphi) (p^N_t)^{1-\mu} + \varphi (p^A_t)^{1-\mu}.
\]

Inflation measures, let $P_t$ denote the aggregate price index. Gross inflation is
\[
\pi_t = \frac{P_t}{P_{t-1}}.
\]
Sectoral inflation rates are
\[
\pi_{N,t} = \frac{P^N_t}{P^N_{t-1}},
\qquad
\pi_{A,t} = \frac{P^A_t}{P^A_{t-1}}.
\]

Gross Domestic Product:

\[
GDP_t = y_t - n_t p^N_t x_t.
\]

\section{Monetary policy}

Monetary policy follows a Taylor rule in logs around the steady state:
\[
\log\left(\frac{R_t}{\bar R}\right)
= \rho\, \log\left(\frac{R_{t-1}}{\bar R}\right)
+ (1-\rho)\left[
\phi_y \log\left(\frac{GDP_t}{\overline{GDP}}\right)
+ \phi_\pi \log(\pi_t)
\right],
\]
where $\bar R$ and $\overline{GDP}$ denote steady state values.

\section{Shock processes}

All exogenous processes follow AR(1) dynamics:
\[
\log e^z_t = \rho_z \log e^z_{t-1} + \sigma_z\, \eta^z_t,
\]
\[
\log e^h_t = \rho_h \log e^h_{t-1} + \sigma_h\, \eta^h_t,
\]
\[
\log e^g_t = \rho_g \log e^g_{t-1} + \sigma_g\, \eta^g_t,
\]
\[
\log e^n_t = \rho_n \log e^n_{t-1} + \sigma_n\, \eta^n_t,
\]
\[
\log e^s_t = \rho_s \log e^s_{t-1} + \sigma_s\, \eta^s_t.
\]

All innovations $\eta^j_t$ are iid standard normal.

\end{document}
